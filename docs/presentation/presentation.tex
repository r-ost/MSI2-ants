\documentclass{beamer}
\usepackage[utf8]{inputenc}
\usepackage[T1]{fontenc}
\usepackage[polish]{babel}
\usepackage{polski}
\usepackage{indentfirst}
\usepackage{hyperref}
\usepackage{graphicx}
\usepackage{float}
\usepackage{amsmath}
\usepackage{xurl}
\usepackage{tikz}
\usepackage[skip=1ex]{caption}
\usepackage{tabularx}
\usepackage{subfig}
\newcolumntype{C}{>{\centering\arraybackslash}X}
\usepackage{subfiles}
\usepackage{listings}
\usepackage{xparse}
\usepackage{svg}

\AtBeginSection[]{}

\author{Jan Szablanowski}
\title{Zastosowanie algorytmu mrówkowego do CVRP - prezentacja wyników}
\date{07.04.2025}
\usepackage{UCAS}
\usepackage{subfiles}
% defs
\def\cmd#1{\texttt{\color{red}\footnotesize $\backslash$#1}}
\def\env#1{\texttt{\color{blue}\footnotesize #1}}
\definecolor{deepblue}{rgb}{0,0,0.5}
\definecolor{deepred}{rgb}{0.6,0,0}
\definecolor{deepgreen}{rgb}{0,0.5,0}
\definecolor{halfgray}{gray}{0.55}

\def\cmd#1{\texttt{\color{red}\footnotesize $\backslash$#1}}
\def\env#1{\texttt{\color{blue}\footnotesize #1}}
\definecolor{deepblue}{rgb}{0,0,1}
\definecolor{deepred}{rgb}{0.6,0,0}
\definecolor{deepgreen}{rgb}{0,0.5,0}
\definecolor{halfgray}{gray}{0.55}

\lstset{
    basicstyle=\ttfamily\small,
    keywordstyle=\bfseries\color{deepblue},
    emphstyle=\ttfamily\color{deepred},    % Custom highlighting style
    stringstyle=\color{deepgreen},
    numbers=left,
    numberstyle=\small\color{halfgray},
    rulesepcolor=\color{red!20!green!20!blue!20},
    frame=shadowbox,
}
\DeclareDocumentCommand \imgsrc { s O {Image credit} +m }{%
  \IfBooleanTF{#1}{%
        \def\tempa{off}%
        \def\tempb{#3}%
        \ifx\tempa\tempb
          \gdef\@imgsrcnavigation{}%
        \else
          \gdef\@imgsrcnavigation{{\tiny #2:\thinspace #3}}%
        \fi
  }{%
        {\tiny #2:\thinspace #3}%
  }%
}

\begin{document}
\beamertemplatenavigationsymbolsempty
\begin{frame}[plain]
    \begin{figure}
        \begin{center}
            \includegraphics[width=0.8\linewidth]{img/logo.png}
        \end{center}
    \end{figure}
    \vspace{-4mm}
    \titlepage
\end{frame}

% \begin{frame}
%     \tableofcontents[sectionstyle=show,subsectionstyle=show/shaded/hide,subsubsectionstyle=show/shaded/hide]
% \end{frame}

\section*{Wprowadzenie}

\begin{frame}{Testowane algorytmy}
  \begin{enumerate}
    \setlength\itemsep{1em}
    \item Algorytm mrówkowy (Ant Colony Optimization - ACO)
    \item Algorytm mrówkowy z heurystyką 2-opt
    \item Algorytm mrówkowy z modyfikacją Max-Min (Max-Min ant system - MMAS)
    \item Algorytm zachłanny (Greedy algorithm)
  \end{enumerate}
\end{frame}

\begin{frame}{Parametry algorytmów ACO i ACO 2-opt}
    % MaxIterations = 100,
    % Alpha = 1.0,
    % Beta = 5.0,
    % EvaporationRate = 0.2,
    % Q = 10,
    % InitialPheromone = 0.1,
    \begin{itemize}
        \setlength\itemsep{1em}
        \item liczba mrówek $=$ liczba wierzchołków
        \item liczba iteracji $= 100$
        \item współczynnik wpływu feromonu $\alpha = 1.0$
        \item współczynnik wpływu heurystyki (odległości) $\beta = 5.0$ 
        \item współczynnik parowania feromonu $EvaporationRate = 0.2$
        \item współczynnik dodawania feromonu $Q = 10$
        \item początkowa ilość feromonu $InitialPheromone = 0.1$
    \end{itemize}
\end{frame}

\begin{frame}{Parametry algorytmu MMAS}
    % PheromoneMin = 0.01,
    % PheromoneMax = pheromoneMax,
    % OnlyBestUpdates = false,
    % StagnationLimit = 20
    \begin{itemize}
        \setlength\itemsep{1em}
        \item minimalna ilość feromonu $PheromoneMin = 0.01$
        \item maksymalna ilość feromonu $PheromoneMax = 10$
        \item początkowa ilość feromonu $InitialPheromone = 10$ 
        \item limit stagnacji (maksymalna liczba iteracji bez poprawy) $StagnationLimit = 20$
    \end{itemize}
\end{frame}

\section*{Hipoteza 1}
\begin{frame}
    \textit{Algorytm mrówkowy z heurystyką 2-opt znajduje trasy o co najmniej 10\% mniejszym koszcie w porównaniu do klasycznego algorytmu mrówkowego.}
\end{frame}

\begin{frame}{Heurystyka 2-opt}
    \begin{itemize}
        \setlength\itemsep{1em}
        \item heurystyka lokalna (ulepsza pojedynczą trasę po jej wyznaczeniu przez kroki algorytmu mrówkowego)
        \item główna idea -- zmiana kolejności odwiedzania klientów w trasie, żeby nie było przecięć tras
        \item heurystyka jest stosowana również w klasycznym TSP
    \end{itemize}
\end{frame}

\begin{frame}{Heurystyka 2-opt -- przykład}
    \begin{figure}
        \centering
        \includegraphics[width=0.5\linewidth]{img/2-opt_wiki.png}
    \end{figure}        
    \imgsrc{PierreSelim, CC BY-SA 3.0, via Wikimedia Commons} 
\end{frame}

\begin{frame}{Wyniki eksperymentów}

\end{frame}

\section*{Hipoteza 2}

\begin{frame}
    \textit{Algorytm mrówkowy z modyfikacją Max-Min znajduje rozwiązanie o koszcie nie większym niż 5\% od kosztu rozwiązania znalezionego przez klasyczny algorytm mrówkowy, ale potrzebuje do tego o 10\% mniejszej liczby pojazdów.}
\end{frame}


\begin{frame}{Heurystyka MAX-MIN}
    \begin{itemize}
        \setlength\itemsep{1em}
        \item zmiana sposobu aktualizacji feromonu
        \item feromon na krawędzi jest aktualizowany tylko przez najlepszą mrówkę w danej iteracji lub globalnie
        \item feromon na krawędzi jest ograniczony przez wartości $PheromoneMin$ i $PheromoneMax$
        \item przy braku poprawy przez pewną liczbę iteracji, algorytm resetuje feromon na krawędziach
        \item celem jest zwiększenie jakości rozwiązań (eksploatacja)
    \end{itemize}
\end{frame}


\begin{frame}{Wyniki eksperymentów}

\end{frame}

\section*{Hipoteza 3}
\begin{frame}
    \textit{Procentowa różnica kosztów wyznaczonych tras między algorytmem zachłannym i klasycznym algorytmem mrówkowym rośnie wraz ze wzrostem liczby klientów na niekorzyść algorytmu zachłannego.}
\end{frame}


\begin{frame}{Wyniki eksperymentów}

\end{frame}


\section*{Hipoteza 4}

\begin{frame}
    \textit{Procentowa różnica kosztów wyznaczonych tras między algorytmem mrówkowym z heurystyką 2-opt i klasycznym algorytmem mrówkowym rośnie wraz ze wzrostem rozproszenia klientów (mierzonego średnią odległością klientów od magazynu) na niekorzyść klasycznego algorytmu.}
\end{frame}


\begin{frame}{Wyniki eksperymentów}

\end{frame}


\begin{frame}[allowframebreaks]
    \frametitle{Bibliografia}
    \nocite{*}
    \bibliographystyle{unsrt}
    \bibliography{../ref.bib}
\end{frame}

\end{document}